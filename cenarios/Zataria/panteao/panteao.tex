\documentclass[letterpaper,twocolumn,openany]{dndbook}

\usepackage[portuguese]{babel}

\usepackage[utf8]{inputenc}
\usepackage[singlelinecheck=false]{caption}
\usepackage{lipsum}
\usepackage{listings}
\usepackage{shortvrb}
\usepackage{stfloats}

\title{Os deuses de Zataria}

\begin{document}
	
	\chapter*{Deuses de Zataria}
	
	\DndDropCapLine{Z}{ataria} é um continente repleto de mitos, magia e mistério. Sendo assim, vários deuses figuram no imaginário dos povos que nele habitam. Aqui estão elencados os mais conhecidos.
	
	\section{Os Dez Admiráveis}
	Dentre os deuses de Zataria, Os Dez Admiráveis são os mais conhecidos. Diz-se que em tempos remotos, eles foram responsáveis pela defesa do mundo ante às forças das trevas.
	
	\subsection{Thelodon: O Inefável}
	Deus do firmamento e das estrelas.
	\par Na cosmogonia é o responsávelpelos alicerces dos céus. 
	\par Representado geralmente como um humanóide velho e de volumosa barba branca, carregando um compasso, com o rosto virado ou meio escondido.
	\par Figura misteriosa, não possui templos e é cultuado esporadicamente em festividades universais, como da passagem do ano.
	\par Não há clérigos que sirvam essa divindade.
	
	\subsection{Kovarus: O Perpétuo}
	Deus dos oceanos, das marés, dos rios, da chuva, de toda água que se move, e dos cavalos.
	\par Na cosmogonia, foi o responsável por fazer com que a lua e o sol iniciassem seu movimento. 
	\par É representado geralmente como um humanóide musculoso, de barba volumosa, com uma coroa e um tridente, por vezes montado em um cavalo, ou em um cavalo marinho.
	\par Popular, costuma ser cultuado por pescadores, marinheiros, e seus templos são comuns em cidades costeiras e ilhas.
	 \par Seus domínios são Conhecimento, Vida e Natureza.
	
	\subsection{Trímata: A Perene}
	Deusa do tempo, da morte, do inverno, e das fronteiras. 
	\par Na cosmogonia, foi responsável por estabelecer os limites entre a terra e os mares.
	\par É representada geralmente como uma mulher usando um véu negro, ou como uma velha também de negro, carregando uma bengala, ou uma ampulheta. Em algumas versões também aparece carregando uma foice.
	\par Trímata é considerada a deusa da "morte boa", que traz o descanso, portanto costumam ser prestados respeitos a ela em funeráis, na esperança que o defunto encontre um caminho tranquilo até a próxima vida, e que seu cadáver não venha a ser reanimado. Também é considerada a padroeira das viúvas.
	\par Seus domínios são Conhecimento, Trapaça e Guerra.
	
	\subsection{Beleren: O Magnânimo}
	Deus da ordem, da justiça e dos muros.
	\par Na cosmogonia, fundou a primeira cidade do mundo e ensinou aos mortais os princípios da civilização.
	\par É representado geralmente como um humanóide musculoso, de barba volumosa, sentado em um trono, segurando um cetro ou um pergaminho.
	\par Muito popular, costuma ser cultuado por militares, burocratas, juízes e pedreiros. Seus templos são comuns em grandes centros urbanos.
	\par Seus domínios são Conhecimento e Tempestade.
	
	\subsection{Glalbor: O Intrépido}
	Deus da guerra e da coragem.
	\par Na cosmogonia, foi o responsável por domar as feras antigas, para que essas não atacassem as cidades dos mortais.
	\par É representado geralmente como um humanóide troncudo, sem barba, por vezes em um cavalo, ou biga, ou ainda, num trenó puxado por sete cães, por vezes em pose de ataque, estocando uma lança.
	\par Popular, costuma ser cultuado por soldados, caçadores de recompensa e aventureiros. Seus templos são comuns em cidade fronteiriças e fortificadas, ou em cidades onde se realizam treinamentos militares.
	\par Sues domínios são Guerra e Natureza.
	
	\subsection{Zarthan: O Magnífico}
	Deus do sol, da luz e das artes.
	\par Na cosmogonia, foi o responsável pela luz do sol e da lua.
	\par É representado geralmente como um humanóide jovem e esbelto, carregando uma harpa.
	\par Popular, costuma ser cultuado por artistas, aristocratas e arquitetos.
	\par Seus domínios são Luz e Trapaça.
	
	\subsection{Vítria: A Sublime}
	Deusa da fertilidade, da alegria e do amor.
	\par Na cosmogonia, foi a responsável por pacificar as forças e espíritos da natureza e permitir o florescimento das primeiras civilizações.
	\par É representada geralmente como uma humanóide jovem e bela, por vezes acompanhada de fadas, por vaezes cercada de pedras e metais preciosos, pérolas e videiras.
	\par Popular, costuma ser cultuada por agricultores, artistas e aristocratas. Seus templos são geralmente construídos no topo de colinas, e as moças que desejam se casar costumam deixar oferendas durante os festivas primaveris.
	\par Seus domínios são Natureza e Trapaça.
	
	\subsection{Polímeto: O Astuto}
	Deus da engenhosidade.
	
	\subparagraph{Subparagraph}
	The subparagraph format with the paragraph indent is likely going to be more familiar to the reader.
	
	\section{Items and Spells}
	The module also includes the functions |\subtitlesection| and |\spellheader| to aid in the proper typesetting of items (including magic items and traps) and spells.
	
	\subtitlesection{Foo's Quill}{Wondrous item, rare}
	This quill has 3 charges. While holding it, you can use an action to expend 1 of its charges. The quill leaps from your hand and writes a contract applicable to your situation.
	
	The quill regains 1d3 expended charges daily at dawn.
	
	\spellheader%
	{Beautiful Typesetting}
	{4th-level illusion}
	{1 action}
	{5 feet}
	{S, M (ink and parchment, which the spell consumes)}
	{Until dispelled}
	You are able to transform a written message of any length into a beautiful scroll. All creatures within range that can see the scroll must make a wisdom saving throw or be charmed by you until the spell ends.
	
	While the creature is charmed by you, they cannot take their eyes off the scroll and cannot willingly move away from the scroll. Also, the targets can make a wisdom saving throw at the end of each of their turns. On a success, they are no longer charmed.
	
	\section{Map Regions}
	The map region functions |\area| and |\subarea| provide automatic numbering of areas.
	
	\area{Village of Hommlet}
	This is the village of hommlet.
	
	\subarea{Inn of the Welcome Wench}
	Inside the village is the inn of the Welcome Wench.
	
	\subarea{Blacksmith's Forge}
	There's a blacksmith in town, too.
	
	\area{Foo's Castle}
	This is foo's home, a hovel of mud and sticks.
	
	\subarea{Moat}
	This ditch has a board spanning it.
	
	\subarea{Entrance}
	A five-foot hole reveals the dirt floor illuminated by a hole in the roof.
	
	\chapter{Text Boxes}
	
	The module has three environments for setting text apart so that it is drawn to the reader's attention. The |quotebox| is used for text that a game master would read aloud.
	
	\begin{quotebox}
		As you approach this module you get a sense that the blood and tears of many generations went into its making. A warm feeling welcomes you as you type your first words.
	\end{quotebox}
	
	\section{As an Aside}
	The other two environments are the |commentbox| and the |paperbox|. The |commentbox| is breakable and can safely be used inline in the text.
	
	\begin{commentbox}{This Is a Comment Box!}
		A |commentbox| is a box for minimal highlighting of text. It lacks the ornamentation of |paperbox|, but it can handle being broken over a column.
	\end{commentbox}
	
	The |paperbox| is not breakable and is best used floated toward a page corner as it is below.
	
	\section{Tables}
	The |dndtable| colors the even rows and is set to the width of a line by default.
	% For more columns, you can say \begin{dndtable}[your options here].
	% For instance, if you wanted three columns, you could say
	% \begin{dndtable}[XXX]. The usual host of tabular parameters are
	% available as well.
	
	\header{Nice table}
	\begin{dndtable}
		\textbf{Table head}  & \textbf{Table head} \\
		Some value  & Some value \\
		Some value  & Some value \\
		Some value  & Some value
	\end{dndtable}
	
	\begin{paperbox}[float=!b]{Behold the Paperbox!}
		The |paperbox| is used as a sidebar. It does not break over columns and is best used with a figure environment to float it to one corner of the page where the surrounding text can then flow around it.
	\end{paperbox}
	
	\chapter{Monsters and NPCs}
	The |monsterbox| environment is used to typeset monster and NPC stat blocks. The module supplies many functions to easily typeset the contents of the stat block
	
	% You can optionally not include the background by saying
	% begin{monsterboxnobg}
	\begin{monsterbox}{Monster Foo}
		\begin{hangingpar}
			\textit{Medium metasyntactic variable (goblinoid), neutral evil}
		\end{hangingpar}
		
		\dndline%
		
		\basics[%
		armorclass = 9 (12 with \emph{mage armor}),
		hitpoints  = \dice{3d8+3},
		speed      = {30 ft., fly 30 ft.},
		]
		
		\dndline%
		
		\stats[
		STR = \stat{12}, % This stat command will autocomplete the modifier for you
		DEX = \stat{8},
		CON = \stat{13},
		INT = \stat{10},
		WIS = \stat{14},
		CHA = \stat{15},
		]
		
		\dndline%
		
		\details[% If you want to use commas in these sections, enclose the description in braces.
		%savingthrows = {Str +0, Dex +0, Con +0, Int +0, Wis +0, Cha +0},
		%skills = {Acrobatics +0, Animal Handling +0, Arcana +0, Athletics +0, Deception +0, History +0, Insight +0, Intimidation +0, Investigation +0, Medicine +0, Nature +0, Perception +0, Performance +0, Persuasion +0, Religion +0, Sleight of Hand +0, Stealth +0, Survival +0},
		%damagevulnerabilities = {},
		%damageresistances = {},
		%damageimmunities = {},
		%conditionimmunities = {},
		senses = {darkvision 60 ft., passive Perception 10},
		languages = {Common, Goblin},
		challenge = {1},
		]
		
		\dndline%
		
		% Traits
		
		\begin{monsteraction}[Innate Spellcasting]
			Foo's spellcasting ability is Charisma (spell save DC 12, +4 to hit with spell attacks). It can innately cast the following spells, requiring no material components:
			\medskip
			\DndInnateSpellLevel{misty step}
			\DndInnateSpellLevel[3]{fog cloud, rope trick}
			\DndInnateSpellLevel[1]{identify}
		\end{monsteraction}
		
		\begin{monsteraction}[Spellcasting]
			Foo is a 3rd-level spellcaster. Its spellcasting ability is Charisma (spell save DC 12, +4 to hit with spell attacks). It has the following sorcerer spells prepared:
			\medskip
			\DndMonsterSpellLevel{blade ward, fire bolt, light, shocking grasp}
			\DndMonsterSpellLevel[1][4]{burning hands, mage armor}
			\DndMonsterSpellLevel[2][2]{scorching ray}
		\end{monsteraction}
		
		\monstersection{Actions}
		\begin{monsteraction}[Multiattack]
			The foo makes two melee attacks.
		\end{monsteraction}
		
		%Default values are shown commented out
		\monsterattack[
		%name=Dagger,
		%enum* type={both,melee,ranged},
		mod=+3,%mod=+0,
		%reach=5,
		%range=20/60,
		%targets=one target,
		dmg=\dice{1d4+1},%dmg=\dice{1d4},
		%dmgtype=piercing,
		%plusdmg=,
		%plusdmgtype=,
		%ordmg=,
		%ordmgwhen=,
		%extra=,
		]
		
		%\monstermelee calls \monsterattack with the melee option
		\monstermelee[
		name=Flame Tongue Longsword,
		mod=+3,%mod=+0,
		%reach=5,
		%targets=one target,
		dmg=\dice{1d8+1},
		dmgtype=slashing,
		ordmg=\dice{1d10+1},
		ordmgwhen=if used with two hands,
		plusdmg=\dice{2d6},
		plusdmgtype=fire
		]
		
		%\monsterranged calls \monsterattack with the ranged option
		\monsterranged[
		name=Assassin's Light Crossbow,
		%mod=+0,
		range=80/320,
		dmg=\dice{1d8},
		dmgtype=piercing,
		extra={, and the target must make a DC 15 Constitution saving throw, taking 24 (7d6) poison damage on a failed save, or half as much damage on a successful one}
		]
	\end{monsterbox}
	
	\chapter{Colors}
	
	\begin{table*}[b]%
		\caption{}\label{tab:colors}
		\header{Colors Supported by This Package}
		
		\begin{dndtable}[XX]
			\textbf{Color}                  & \textbf{Description} \\
			|PhbLightGreen|                 & Light green used in PHB Part 1 (Default) \\
			|PhbLightCyan|                  & Light cyan used in PHB Part 2 \\
			|PhbMauve|                      & Pale purple used in PHB Part 3 \\
			|PhbTan|                        & Light brown used in PHB appendix \\
			|DmgLavender|                   & Pale purple used in DMG Part 1 \\
			|DmgCoral|                      & Orange-pink used in DMG Part 2 \\
			|DmgSlateGray| (|DmgSlateGrey|) & Blue-gray used in PHB Part 3 \\
			|DmgLilac|                      & Purple-gray used in DMG appendix \\
		\end{dndtable}
	\end{table*}
	
	This package provides several global color variables to style |commentbox|, |quotebox|, |paperbox|, and |dndtable| environments.
	
	\header{Box Colors}
	\begin{dndtable}[lX]
		\textbf{Color}    & \textbf{Description} \\
		|commentboxcolor| & |commentbox| background \\
		|paperboxcolor|   & |paperbox| background \\
		|quoteboxcolor|   & |quotebox| background \\
		|tablecolor|      & background of even |dndtable| rows \\
	\end{dndtable}
	
	They also accept an optional color argument to set the color for a single instance. See Table~\ref{tab:colors} for a list of core book accent colors.
	
	\begin{lstlisting}
	\begin{dndtable}[cX][PhbLightCyan]
	\textbf{d8} & \textbf{Item} \\
	1           & Small wooden button \\
	2           & Red feather \\
	3           & Human tooth \\
	4           & Vial of green liquid \\
	6           & Tasty biscuit \\
	7           & Broken axe handle \\
	8           & Tarnished silver locket \\
	\end{dndtable}
	\end{lstlisting}
	
	\begin{dndtable}[cX][PhbLightCyan]
		\textbf{d8} & \textbf{Item} \\
		1           & Small wooden button \\
		2           & Red feather \\
		3           & Human tooth \\
		4           & Vial of green liquid \\
		6           & Tasty biscuit \\
		7           & Broken axe handle \\
		8           & Tarnished silver locket \\
	\end{dndtable}
	
	\section{Themed Colors}
	Use |\\setthemecolor[<color>]| to set |themecolor|, |commentcolor|, |paperboxcolor|, and |tablecolor| to a specific color. Calling |\\setthemecolor| without an argument sets those colors to the current |themecolor|. In the following example the group limits the change to just a few boxes; after the group finishes, the colors are reverted to what they were before the group started.
	
	\begin{lstlisting}
	\begingroup
	\setthemecolor[PhbMauve]
	
	\begin{commentbox}{This Comment Is in Mauve}
	This comment is in the the new color.
	\end{commentbox}
	
	\begin{paperbox}{This Sidebar Is Also Mauve}
	The sidebar is also using the new theme color.
	\end{paperbox}
	\endgroup
	\end{lstlisting}
	
	\begingroup
	\setthemecolor[PhbMauve]
	
	\begin{commentbox}{This Comment Is in Mauve}
		This comment is in the the new color.
	\end{commentbox}
	
	\begin{paperbox}{This Sidebar Is Also Mauve}
		The sidebar is also using the new theme color.
	\end{paperbox}
	
	\endgroup
	
\end{document}
