\documentclass[letterpaper,twocolumn,openany]{dndbook}

\usepackage[portuguese]{babel}

\usepackage[utf8]{inputenc}
\usepackage[singlelinecheck=false]{caption}
\usepackage{lipsum}
\usepackage{listings}
\usepackage{shortvrb}
\usepackage{stfloats}

\title{Os deuses de Zataria}

\begin{document}
	
	\chapter*{Deuses de Zataria}
	
	\DndDropCapLine{Z}{ataria} é um continente repleto de mitos, magia e mistério. Sendo assim, vários deuses figuram no imaginário dos povos que nele habitam. Aqui estão elencados os mais conhecidos.
	
	\section{Os Admiráveis}
	Dentre os deuses de Zataria, Os Admiráveis são os mais conhecidos. Diz-se que em tempos remotos, eles foram responsáveis pela defesa do mundo ante às forças das trevas.
	
	\subsection{Thelodon: O Inefável}
	Deus do firmamento e das estrelas.
	\par Na cosmogonia é o responsável pelos alicerces dos céus. 
	\par Representado geralmente como um humanóide velho e de volumosa barba branca, carregando um compasso, com o rosto virado ou meio escondido.
	\par Figura misteriosa, não possui templos e é cultuado esporadicamente em festividades universais, como da passagem do ano.
	\par Não há clérigos que sirvam essa divindade.
	
	\subsection{Kovarus: O Perpétuo}
	Deus dos oceanos, das marés, dos rios, da chuva, de toda água que se move, e dos cavalos.
	\par Na cosmogonia, foi o responsável por fazer com que a lua e o sol iniciassem seu movimento. 
	\par É representado geralmente como um humanóide musculoso, de barba volumosa, com uma coroa e um tridente, por vezes montado em um cavalo, ou em um cavalo marinho.
	\par Popular, costuma ser cultuado por pescadores, marinheiros, e seus templos são comuns em cidades costeiras e ilhas.
	 \par Seus domínios são Conhecimento, Vida e Natureza.
	
	\subsection{Trímata: A Perene}
	Deusa do tempo, da morte, do inverno, e das fronteiras. 
	\par Na cosmogonia, foi responsável por estabelecer os limites entre a terra e os mares.
	\par É representada geralmente como uma mulher usando um véu negro, ou como uma velha também de negro, carregando uma bengala, ou uma ampulheta. Em algumas versões também aparece carregando uma foice.
	\par Trímata é considerada a deusa da "morte boa", que traz o descanso, portanto costumam ser prestados respeitos a ela em funeráis, na esperança que o defunto encontre um caminho tranquilo até a próxima vida, e que seu cadáver não venha a ser reanimado. Também é considerada a padroeira das viúvas.
	\par Seus domínios são Conhecimento, Trapaça e Guerra.
	
	\subsection{Beleren: O Magnânimo}
	Deus da ordem, da justiça e dos muros.
	\par Na cosmogonia, fundou a primeira cidade do mundo e ensinou aos mortais os princípios da civilização.
	\par É representado geralmente como um humanóide musculoso, de barba volumosa, sentado em um trono, segurando um cetro ou um pergaminho.
	\par Muito popular, costuma ser cultuado por militares, burocratas, juízes e pedreiros. Seus templos são comuns em grandes centros urbanos.
	\par Seus domínios são Conhecimento e Tempestade.
	
	\subsection{Glalbor: O Intrépido}
	Deus da guerra e da coragem.
	\par Na cosmogonia, foi o responsável por domar as feras antigas, para que essas não atacassem as cidades dos mortais.
	\par É representado geralmente como um humanóide troncudo, sem barba, por vezes em um cavalo, ou biga, ou ainda, num trenó puxado por sete cães, por vezes em pose de ataque, estocando uma lança.
	\par Popular, costuma ser cultuado por soldados, caçadores de recompensa e aventureiros. Seus templos são comuns em cidade fronteiriças e fortificadas, ou em cidades onde se realizam treinamentos militares.
	\par Sues domínios são Guerra e Natureza.
	
	\subsection{Zarthan: O Magnífico}
	Deus do sol, da luz e das artes.
	\par Na cosmogonia, foi o responsável pela luz do sol e da lua.
	\par É representado geralmente como um humanóide jovem e esbelto, carregando uma harpa.
	\par Popular, costuma ser cultuado por artistas, aristocratas e arquitetos.
	\par Seus domínios são Luz e Trapaça.
	
	\subsection{Vítria: A Sublime}
	Deusa da fertilidade, da alegria e do amor.
	\par Na cosmogonia, foi a responsável por pacificar as forças e espíritos da natureza e permitir o florescimento das primeiras civilizações.
	\par É representada geralmente como uma humanóide jovem e bela, por vezes acompanhada de fadas, por vezes cercada de pedras e metais preciosos, pérolas e videiras.
	\par Popular, costuma ser cultuada por agricultores, artistas e aristocratas. Seus templos são geralmente construídos no topo de colinas, e as moças que desejam se casar costumam deixar oferendas durante os festivas primaveris.
	\par Seus domínios são Natureza e Trapaça.
	
	\subsection{Polímeto: O Astuto}
	Deus do conhecimento, da invenção e da escrita.
	\par Na cosmogonia, ele foi o responsável por conceder o dom da escrita ao mortais, bem como o primeiro a lhes transmitir os conhecimentos a respeito de como tirar proveito das forças da natureza.
	\par É representado geralmente como um humanóide esguio e de baixa estatura, geralmente segurando uma pena em sua mão direita e um pergaminho em sua mão esquerda.
	\par Popular, costuma ser cultuado por ferreiros, construtores e artistas.
	\par Seus domínios são Conhecimento e Trapaça.
	
	\subsection{Cerília: A Insondável}
	Deusa da lua, das estações e dos sonhos.
	\par Na cosmogonia, ela quem concedeu aos seres vivos a dádiva do sono, para que pudessem descansar.
	\par Ela também foi quem ordenou as fases da lua, as estações do ano, bem como muitos dos ciclos da natureza.
	\par É representada geralmente como uma humanóide jovem, usando uma túnica e segurando uma varinha.
	\par Seus domínios são Conhecimento e Natureza.
	
	\subsection{Thelínea: A Perfeita}
	Deusa da terra, das plantas e das colheitas.
	\par Na cosmogonia, ensinou a agricultura aos mortais.
	\par É representada geralmente como uma humanóide robusta, segurando um faixo de cereais em suas mãos.
	\par Seus domínios são Vida e Natureza.
	
	\section{Os Inomináveis}
	Se Os Admiráveis foram os responsáveis pela defesa do mundo em tempos longínquos, Os Inomináveis podem ser ditos aqueles que o ameaçavam. Considerados deuses das trevas, apenas criaturas sinistras e cultistas insandecidos ousam prestar respeito a essas figuras nefastas.
	
	\subsection{Thefarion: Os Gêmeos do Vazio}
	Senhor do Vazio, Thefarion é uma criatura com duas cabeças: Járion e Férudras, gêmeos distintos entre si, porém presos um ao outro.
	\par Járion é uma existência alheia ao sentimento ou empatia. Seu desejo é desmantelar o universo, para que possa compreendê-lo. Para ele, todos os meios são justificados na sua busca pelo conhecimento supremo.
	\par Férudas, por outro lado, deseja poder absoluto sobre todas as coisas. Todas as vontades devem se dobrar à sua.
	\par Dizem algumas lendas que não fosse o eterno conflito entre os dois, eles já teriam alcançado seus objetivos nefastos.
	
	\subsection{Balarion: O Inexorável}
	Senhor da Opressão, Balarion não possui limites na sua busca por poder e domínio.
	\par Servo de Férudras, compartilha o desejo de seu senhor pelo poder sobre todas as coisas.
	\par Na Era de Ouro, quando os deuses em seus primórdios travaram seus conflitos, Balarion era um adversário incansável, até ter sido ferido mortalmente por Kovarus.
	\par Depois disso, Balarion perdeu seu vigor ilimitado, porém se tornou ainda mais ardiloso.
	
	\subsection{Sarvath: A Imperdoável}
	Senhora da Extinção, Sarvath nutri um ódio extremo por todos os mortais, a quem ela considera inferiores, e indignos das dádivas da natureza.
	\par Serva de Járion, buscas assim como seu senhor o conhecimento absoluto, que ela acredita ser a chave para posse de todas as coisas.
	
	\subsection{Osgoth: O Devorador}
	Senhor da Miséria, Osgoth possui fome e sede que nunca se saciam.
	\par Dizem alguns mitos que Osgoth tragou planos inteiros, ele e suas hordas de devoradores banqueteando-se da carne e embebedando-se do sangue de suas vítimas.
	\par Segundo uma lenda, foi enganado por Zarthan para que entrasse numa caixa feita por Beleren, na qual foi aprisionado, e lançado nas profundezas de um abismo.
	
	\subsection{Tanath: O Sanguinário}
	Senhor da Matança, Tanath regojiza-se apenas na violência.
	\par Segundo lendas sua ira implacável levou-o a devastar incontáveis planos junto de seus exércitos incansáveis.
	
	\subsection{Megiliath: O Tirano}
	Senhor da Opulência, Megiliath deseja se elevar acima de tudo e de todos.
	\par Segundo o mito, Megiliath destruía todos aqueles que recusavam curvar-se diante dele, até que Zarthan o amarrou com as cordas que Polímeto trançou dos cabelos de Vítria e lançou-o nas profundezas de um oceano.
	
	\subsection{Tamariel: A Obstinada}
	Senhora dos Desejos, Tamariel não tem limites na sua busca por poder, glória, riqueza e prazer.
	\par Segundo as lendas, muitos planos ruíram por seus soberanos se deixaram seduzir pelas promessas de Tamariel.
	
	\subsection{Aziloch: O Sinistro}
	Senhor do Sacrilégio, Aziloch busca poder através do domínio dos segredos mais nefastos da criação.
	\par Ávido por conhecimento e poder, Aziloch está sempre realizando experimentos perversos, corrompendo a criação com sua cruel engenhosidade.
	\par Suas criações terríveis teriam sido as precursoras de todas as abominações e monstruosidades que assolam os planos.
	
	\subsection{Ciliath: A Ensandecida}
	Senhora da Loucura, Ciliath corrompia as mentes de todos que entravam em sua presença.
	\par Com seus poderes profanos, era capaz de distorcer a realidade, tornando em realidade os pesadelos do mortais. Diz-se que essa é a origem de muitas criaturas macabras.
	\par Para ela a loucura era o estágio supremo da existência, e seu desejo é extender sua loucura a todas as mentes do universo.
	
	\subsection{Nagith: A Mortífera}
	Senhora da Corrupção, Nagith causava desgraça e destruição por onde quer que passasse.
	\par Segundo os mitos, ela é a origem das doenças, da necromancia e dos venenos.
	\par Para ela a vida era um estado de imperfeição, que só podia ser corrigida com a mortas.
	\par Suas hordas de mortos vivos assolaram os planos por milênios, até que finalmente ela teria sido aprisionada num caixão de ferro celeste, e enterrada nas profundezas da terra.
	
\end{document}
