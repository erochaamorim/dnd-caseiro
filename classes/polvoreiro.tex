\documentclass[letterpaper,twocolumn,openany]{dndbook}

\usepackage[portuguese]{babel}

\usepackage[utf8]{inputenc}
\usepackage[singlelinecheck=false]{caption}
\usepackage{lipsum}
\usepackage{listings}
\usepackage{shortvrb}
\usepackage{stfloats}
\usepackage{multirow}
\usepackage[absolute,overlay]{textpos}

\newcolumntype{L}[1]{>{\hsize=#1\hsize\raggedright\arraybackslash}X}%
\newcolumntype{R}[1]{>{\hsize=#1\hsize\raggedleft\arraybackslash}X}%
\newcolumntype{C}[1]{>{\hsize=#1\hsize\centering\arraybackslash}X}%

\begin{document}
	\section{Polvoreiro}
	O rosto escondido sob a aba do chapéu, um rifle apoiado no ombro, e um revólver no coldre, pronto para ser sacado a qualquer momento.
	
	O Polvoreiro é um combatente rápido e letal, treinado no uso da pólvora, seja ele um especialista em pistolas, rifles, ou explosivos.

	\subsection{Pólvora: A arma do futuro}
	Bem mais rápida que um feitiço, e bem mais potente que uma flecha, a pólvora está se tornando cada vez mais comum, bem como as máquinas a vapor.
	
	Apesar de ser difícil encontrar essas armas, você por algum motivo teve contato com elas e aprendeu a usá-las. Muitos dos seus oponentes serão pegos de surpresa quando de repente o seu "bastão de ferro" começar a "atirar raios".

	\header{Polvoreiro}
	{\footnotesize
	\begin{dndtable}[R{.5} C{.5} L{2}]
		\textbf{Nível} & \textbf{Bônus de Proficiência} & \textbf{Características} \\
		1 & +2 & Estilo de Luta, Recarga Rápida \\
		2 & +2 & Adrenalina \\
		3 & +2 & Especialidade de Polvoreiro \\
		4 & +2 & Melhoria de Atributo \\
		5 & +3 & Habilidade Especializada de Polvoreiro \\
		6 & +3 & Melhoria de Atributo \\
		7 & +3 & Habilidade Especializada de Polvoreiro \\
		8 & +3 & Melhoria de Atributo \\
		9 & +4 & Indomável \\
		10 & +4 & Habilidade Especializada de Polvoreiro \\
		11 & +4 & Habilidade Especializada de Polvoreiro \\
		12 & +4 & Melhoria de Atributo \\
		13 & +5 & Indomável (2x) \\
		14 & +5 & Melhoria de Atributo \\
		15 & +5 & Habilidade Especializada de Polvoreiro \\
		16 & +5 & Melhoria de Atributo \\
		17 & +6 & Adrenalina (2x), Indomável (3x) \\
		18 & +6 & Habilidade Especializada de Polvoreiro \\
		19 & +6 & Melhoria de Atributo \\
		20 & +6 & Habilidade Especializada de Polvoreiro \\
	\end{dndtable}
	}
	
	\subsection{Aventureiros Intrépidos}
	Num mundo dominado pela espada e pelo machado, apenas os mais valentes, ou os mais loucos, se arriscariam a mexer com pólvora.
	
	A explosiva e perigosa cocção parece infectar os polvoreiros com um humor semelhante.	São geralmente indivíduos diretos e corajosos, que não costumam economizar munição quando é preciso atirar.
	
	\section{Características da Classe}
	Como um polvoreiro, você possui as seguintes características de classe.
	
	\subsubsection{Pontos de Vida}
	\noindent\textbf{Dado de Vida:} 1d10 por nível de polvireiro \\
	\noindent\textbf{Pontos de Vida no 1º Nível:} 10 + seu modificador de Constituição \\
	\noindent\textbf{Pontos de Vida em Níveis Avançados:} 1d10 (ou 6) + seu modificador de Constituição por cada nível de doutor da praga \\
	
	\subsubsection{Proficiências}
	\noindent\textbf{Armadura:} Armaduras leves \\
	\noindent\textbf{Armas:} Pistolas, escopetas, rifles e bombas \\
	\noindent\textbf{Ferramentas:} Nenhuma \\
	\noindent\textbf{Testes de Resistência:} Destreza, Constituição \\
	\noindent\textbf{Perícias:} Escolha duas dentre Acrobracia, Atletismo, História, Discernimento, Intimidação, Percepção, Sobrevivência e Prestidigitação\\
	
	\subsection{Estilo de Luta}
	Você adota um estilo particular em combate que é sua especialidade. Escolha uma das opções a seguir. Você não pode adotar um estilo mais de uma vez, mesmo que receba a opção de escolher um estilo de luta mais de uma vez.
	
	\subsubsection{Mira na Mosca}
	Você ganha um bônus de +2 nas suas rolagens de ataques com armas à distância.
	
	\subsubsection{Duelista}
	Você ganha um bônus de +2 nas suas rolagens de dano quando você estiver utilizando uma pistola em uma mão e nenhuma outra arma.
	
	\subsubsection{Atirar pra Matar}
	Quando você rolar 1 ou 2 num dado de dano para um ataque que você fez com um rifle ou escopeta, você pode re-rolar esse dado. Se escolhe re-rolar, você é obrigado a usar o novo valor mesmo se for 1 ou 2.
	
	\subsubsection{Cobertura}
	Quando uma criatura que você pode ver ataca outro alvo que não seja você, e esteja dentro do alcance padrão da sua arma, você pode usar sua reação para impor uma desvantagem da rolagem de ataque. Você tem que estar usando uma pistola ou rifle para usar essa habilidade.
	
	\subsubsection{À Mexicana}
	Quando você estiver usando duas pistolas, pode adicionar seu modificador de destreza no dano do segundo ataque.
	
	\subsection{Recarga Rápida}
	Sua grande maestria no uso de armas de fogo permite que você recarregue a arma que esteja empunhando usando apenas uma ação bônus.
	
	\subsection{Adrenalina}
	A partir 2º nível, seu sangue ferve quando você está no meio da ação. No seu turno, você pode escolher executar uma ação adicional. Uma vez que você utilize essa habilidade, você deve completar um descanso longo ou curto antes de utilizá-la novamente.
	\par A partir do 17º nível, você pode utilizá-la duas vezes antes de descansar, mas apenas uma vez por turno.
	
	\subsection{Especialidade de Polvoreiro}
	No 3º nível você passa a voltar sua atenção a um tipo de armamento específico, e a desenvolver habilidades e técnicas de combate em torno dele. Escolha Franco-Atirador, Pistoleiro ou Granadeiro, especialidades descritas em detalhe na seção "Especialidades de Polvoreiro", no fim da descrição desta classe. A sua especialidade te concede habilidades e melhorias no 3º nível e novamente no 5º, no 7º, no 10º, no 11º, no 15º no 18º, e no 20º nível.
	
	\subsection{Indomável}
	A partir do 9º nível você pode escolher re-rolar um teste de resistência que você falhar. Se escolher fazê-lo você deve usar o novo resultado e não pode usar esta habilidade novamente antes de completar um descanso completo.
	\par Você pode utilizar essa habilidade duas vezes entre descansos a partir do 13º nível e três vezes a partir do 17º nível.
	
	\section{Especialidades de Polvoreiro}
	Diferentes polvoreiros escolhem diferentes abordagens para aperfeiçoar suas habilidades. A especialidade escolhida por você reflete sua abordagem.
	
	\subsection{Franco-Atirador}
	A especialidade do Franco-Atirador é dominar o campo de batalha com suas armas de longo alcance. Os polvoreiros que escolhem essa especialidade fazem do rifle sua arma principal, e raramente seus inimigos conseguem chegar perto o suficiente.
	
	\subsubsection{Manter Distância}
	A partir de quando você adota essa especialidade no 3º nível, você pode usar as ações \textit{Desengajar} e \textit{Correr} com uma ação bônus.
	
	\subsubsection{Tiro Rápido}
	A partir do 5º nível, você pode utilizar sua ação e sua ação bônus para disparar dois tiros seguidos com um rifle. Você consome munição por cada ataque normalmente.
	
	\subsubsection{Alvejar}
	A partir do 7º nível, no início de uma rodada de combate, você pode declarar um alvo. Se o fizer, você recebe um bônus de +5 nas rolagens de acerto e dano no seu próximo ataque contra o alvo. Se realizar um ataque com esse bônus, você não poderá se mover ou realizar outras ações no seu turno. Você não pode usar essa habilidade caso você tenha se movido no seu turno anterior.
	
	\subsubsection{Atirador de Elite}
	A partir do 10º nível, suas rolagens de ataque com rifles são tratadas como acerto crítico se o valor do d20 for 18, 19 ou 20.
	
	\subsubsection{Neutralizar}
	A partir do 11º nível, quando uma criatura de tamanho Grande ou menor for atingida por um ataque beneficiado pela sua habilidade de \textit{Alvejar}, ela deve realizar um teste de resistência Constituição de CD = 8 + seu bônus de proficiência + seu modificador de destreza. Se falhar, essa criatura não poderá realizar ações ou se mover em seu próximo turno.
	
	\subsubsection{Percepção Balística}
	A partir do 15º nível, todos os ataques à distância contra você tem desvantagem, desde que você possa ver o atacante. Quando você usar usar a ação \textit{Esquivar}, você pode adicionar seu bônus de proficiência em sua CA até o final do turno.
	
	\subsubsection{Chumbo Grosso}
	A partir do 18º nível, você pode rolar os dados de dano três vezes quando realizar um acerto crítico. Além disso, você pode adicionar seu bônus de proficiência na rolagem de dano de todos os seus ataques com rifles.
	
	\subsubsection{Olhos de Águia}
	A partir do 20º nível seu valor de Sabedoria aumenta em +1 (podendo passar o limite de 20, até 21).
	Você ganha proficiência em testes de Percepção. Se já possuir, ganha proficiência dupla.
	Uma vez por turno, você pode adicionar o seu modificador de Sabedoria na rolagem de ataque ou de dano de um de seus ataques contra uma criatura que você possa ver e que não tenha cobertura.
	Você pode escolher usar essa habilidade depois da rolagem, mas antes de que os efeitos da rolagem sejam aplicados.
	
	
\end{document}
