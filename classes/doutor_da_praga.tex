\documentclass[letterpaper,twocolumn,openany]{dndbook}

\usepackage[portuguese]{babel}

\usepackage[utf8]{inputenc}
\usepackage[singlelinecheck=false]{caption}
\usepackage{lipsum}
\usepackage{listings}
\usepackage{shortvrb}
\usepackage{stfloats}
\usepackage{multirow}
\usepackage[absolute,overlay]{textpos}
\usepackage[colorlinks=true, linkcolor=black]{hyperref}

\newcolumntype{L}[1]{>{\hsize=#1\hsize\raggedright\arraybackslash}X}%
\newcolumntype{R}[1]{>{\hsize=#1\hsize\raggedleft\arraybackslash}X}%
\newcolumntype{C}[1]{>{\hsize=#1\hsize\centering\arraybackslash}X}%

\begin{document}
	\section{Doutor da Praga}
	Um chapéu escuro cobrindo a cabeça, uma máscara bicuda escondendo seu rosto. A capa, luvas e botas, completam o serviço de tornar a figura completamente enigmática e anônima.
	
	O Doutor da Praga é uma figura sinistra, e quase sempre sua presença é um mal sinal. Isso é claro, na mente dos ignorantes que nada sabem a respeito da Ciência e mais particularmente da Medicina.
	
	No meio de um mundo tomado pela magia, superstição e  ignorância, você é um dos poucos dotados de uma dose saudável de ceticismo, e também de perspicácia o bastante para explorar a miríade de segredos que a anatomia do corpo mantém trancados a sete chaves.
	
	Claro, devemos admitir que nem todos os doutores tem boas intenções. Pesquisas nem sempre tem um propósito nobre, e os métodos utilizados nem sempre podem ser considerados "éticos". Mas mesmo assim, não se pode culpar a Ciência pelo uso que se faz dela.
	
	Além disso, nem mesmo os bons doutores costumam fazer o tipo convencional. Coisas estranhas podem acontecer quando se passa tempo o bastante dissecando cadáveres e misturando reagentes num laboratório escuro.
	
	\subsection{Poder da Ciência}
	Os doutores da praga acreditam no potencial da inteligência humana acima de tudo. E para eles, a forma mais eficaz de desenvolver esse potencial é através da ciência.
	
	Suas habilidades podem parecer sobrenaturais ou mágicas para alguns, mas para eles, tudo pode ser explicado em termos científicos, mesmo que nem todos sejam capazes de compreender.
	
	Alguns doutores da praga compartilham uma curiosidade também pelos conhecimentos arcanos, enquanto outros acham que se trata de uma perda de tempo. Outros acreditam que a magia é para poucos, e que a ciência é muito mais apropriada para a maioria das mentes. Outros ainda, acreditam que magia e ciência são dois lados da mesma moeda, abordagens diferentes para compreender o mundo.
	
	De toda forma, os doutores da praga preferem colocar suas fichas nas explicações científicas do que na superstição transmitida pelos seus ancestrais ignorantes.

	\onecolumn
	\begin{textblock*}{5cm}(15.45cm, 1.4cm) % {block width} (coords)
		\footnotesize \sffamily \bfseries -Cocções Diárias por Nível-
	\end{textblock*}
	\header{Doutor da Praga}
	{\footnotesize
	\begin{dndtable}[R{0.25} C{1.5} L{3} C{1.5} R{0.25} R{0.25} R{0.25} R{0.25} R{0.25} R{0.25} R{0.25} R{0.25} R{0.25}]
		\label{tab:doutor_da_praga}
		\textbf{Nível} & \textbf{Bônus de Proficiência} & \textbf{Características} & \textbf{Pílulas Conhecidas} & \textbf{1º} & \textbf{2º} & \textbf{3º} & \textbf{4º} & \textbf{5º} \\
		1  & +2 & Letra de Médico, Farmacologia  & 2 & 2 & - & - & - & - \\
		2  & +2 & Especialização Médica  & 2 & 2 & - & - & - & - \\
		3  & +2 & - & 2 & 2 & 1 & - & - & - \\
		4  & +2 & Melhoria de Atributo & 2 & 3 & 1 & - & - & - \\
		5  & +3 & Especialidade Médica & 2 & 3 & 1 & 1 & - & - \\
		6  & +3 & - & 3 & 3 & 2 & 1 & - & - \\
		7  & +3 & Especialidade Médica & 3 & 4 & 2 & 1 & 1 & - \\
		8  & +3 & Melhoria de Atributo & 3 & 4 & 2 & 2 & 1 & - \\
		9  & +4 & Especialidade Médica & 3 & 4 & 2 & 2 & 1 & 1 \\
		10 & +4 & - & 3 & 4 & 3 & 2 & 2 & 1 \\
		11 & +4 & Especialidade Médica & 4 & 4 & 3 & 3 & 2 & 1 \\
		12 & +4 & Melhoria de Atributo & 4 & 4 & 3 & 3 & 2 & 2 \\
		13 & +5 & Especialidade Médica & 4 & 4 & 3 & 3 & 3 & 2 \\
		14 & +5 & - & 4 & 5 & 3 & 3 & 3 & 2 \\
		15 & +5 & Especialidade Médica & 4 & 5 & 3 & 3 & 3 & 3 \\
		16 & +5 & Melhoria de Atributo & 5 & 5 & 3 & 3 & 3 & 3 \\
		17 & +6 & Especialidade Médica & 5 & 5 & 3 & 3 & 3 & 3 \\
		18 & +6 & - & 5 & 5 & 4 & 3 & 3 & 3 \\
		19 & +6 & Melhoria de Atributo & 5 & 5 & 4 & 3 & 3 & 3 \\
		20 & +6 & Especialidade Médica & 5 & 5 & 4 & 3 & 3 & 3 \\
	\end{dndtable}
	}
	\twocolumn
	
	\subsection{Apoteose Antrópica}
	Os doutores da praga acreditam na Ciência como um meio dos mortais deixarem para trás as mazelas que os afligem: a fome, a peste e a velhice. Para esse fim, basta que a Ciência avance o suficiente.
	
	É melhor do que fazer barganhas com seres sombrios ou desperdiçar os melhores dias da vida rogando súplicas a uma estátua glorificada qualquer.
	
	\section{Características da Classe}
	Como um doutor da praga, você possui as seguintes características de classe.
	
	\subsubsection{Pontos de Vida}
	\noindent\textbf{Dado de Vida:} 1d6 por nível de doutor da praga \\
	\noindent\textbf{Pontos de Vida no 1º Nível:} 6 + seu modificador de Constituição \\
	\noindent\textbf{Pontos de Vida em Níveis Avançados:} 1d6 (ou 4) + seu modificador de Constituição por cada nível de doutor da praga \\
	
	\subsubsection{Proficiências}
	\noindent\textbf{Armadura:} Nenhuma \\
	\noindent\textbf{Armas:} Adagas, dardos, estilingues, cajados, bestas leves e redes \\
	\noindent\textbf{Ferramentas:} Kit de Medicina, Kit de Farmacologia \\
	\noindent\textbf{Testes de Resistência:} Inteligência, Sabedoria \\
	\noindent\textbf{Perícias:} Você automaticamente recebe a perícia Medicina e escolha mais uma entre Lidar com Animais, Arcana, História, Investigação, Natureza e Sobrevivência \\
	
	\subsection{Letra de Médico}
	Você é capaz de escrever rapidamente, e num estilo que apenas doutores da praga são capazes de ler com clareza. Qualquer outro indivíduo que tente ler algo escrito por você em letra de médico deve passar num teste de Inteligência de CD: 8 + seu modificador de proficiência + seu modificador de inteligência, e não será capaz de extrair os detalhes da informação, apenas conseguirá ler por altos.
	Caso o indivíduo falhe, ele não poderá fazer uma nova tentativa para decifrar o que está escrito.
	
	\subsection{Farmacologia}
	Como um estudante da farmacologia, você tem um livro de fórmulas para o preparo de substâncias com as mais diversas aplicações.
	
	\subsubsection{Pílulas}
	No 1º nível, você conhece dois tipos de pílulas. Em níveis mais avançados você aprende novos tipos de pílula à sua escolha, conforme a \hyperref[tab:doutor_da_praga]{\textbf{tabela}} da classe. Consulte a seção \textbf{\nameref{sec:pilulas}} para mais detalhes.
	
	\begin{paperbox}[float=!b]{Seu Livro de Fórmulas}
		\label{box:seu_livro_de_formulas}
		As cocções presentes no seu livro de fórmula à medida que você avança níveis representam o resultado dos seus esforços de pesquisa, bem como os avanços intelectuais que você teve a respeito das leis da natureza. É possível que você encontre outras fórmulas em suas aventuras. Você poderia encontrar fórmulas de cocções no livro de outro doutor da praga, por exemplo, ou em anotações misteriosos encontradas num velho baú em um laboratório subterrâneo abandonado.
		\subparagraph{Copiando uma Fórmula pra seu Livro} Quando você encontrar a fórmula de uma cocção de 1º nível ou superior, você pode copiá-la para seu livros de fórmulas caso ela seja de um nível de cocção que você é capaz de produzir, e se você tiver tempo e recursos o bastante para decifrá-la e copiá-la.
		\par Copiar uma fórmula para seu livro de fórmulas envolve reproduzir a forma básica da fórmula e decifrar o sistema de notação que foi usado pelo doutor da praga que a escreveu. Você tem que praticar a cocção da fórmula até que você compreenda os passos e cuidados necessários com os ingredientes, e então transcrever isso para seu livro de fórmulas usando seu próprio sistema de notação.
		\par Para cada nível da cocção, o processo leva 2 horsa e custa 50 po. O custo representa os ingredientes que você gasta durante os experimentos até dominar a fórmula, bem como a tinta necessária para as anotações. Uma vez que você tenha gasto o tempo e o dinheiro, você poderá preparar a cocção normalmente, como faz com suas demais cocções conhecidas.
		\subparagraph{Substituindo o Livro}
		Você pode copiar uma fórmula do seu livro de fórmulas para outro - por exemplo, se você quiser fazer uma cópia de segurança do seu livro. É como copiar uma fórmula nova para seu livro de fórmulas, exceto que é mais fácil e mais rápido, já que você já conhece a fórmula e compreende seu próprio sistema de notação. Você só precisa gastar 1 hora e 10 po para cada nível da fórmula copiada.
		\par Se você perder seu livro de fórmulas, você pode usar o mesmo procedimento para transcrever as fórmulas das cocções que você tem preparadas para um novo livro de fórmulas. Preencher o resto do livro de fórmulas exigirá que você adquira novas fórmulas pelos meios tradicionais. Por essa razão, muitos doutores da praga costumam manter guardado um livro de fórmulas reserva em um local seguro.
		\subparagraph{A Aparência do Livro} Seu livro de fórmulas é uma compilação única de fórmulas, com suas próprias decorações e notas de rodapé. Pode ser um comum e funcional volume de capa dura que você ganhou de seu mestre, ou um elegante tomo de luxo que você encontrou numa livraria antiga, ou até mesmo uma coleção mal organizada de folhas com anotações que você juntou depois de perder seu antigo livro num acidente.
	\end{paperbox}

	\subsubsection{Livro de Fórmulas}
	No primeiro nível você possui um livro de fórmulas contendo três fórmulas de cocções de 1º nível à sua escolha.
	
	\subsubsection{Preparando e Utilizando Cocções}
	A \hyperref[tab:doutor_da_praga]{\textbf{tabela}} de Doutor da Praga mostra quantas cocções você pode preparar e utilizar por dia. Para utilizar uma dessas cocções você deve gastar um dos seus usos diário do mesmo nível da cocção ou superior. Você recupera todos os seus usos após um descanso longo.
	\par Você prepara uma lista de cocções que estão disponíveis para uso. Para fazê-lo escolha um número de cocções de seu Livro de Fórmulas igual a seu modificador de inteligência + seu nível de doutor da praga dividido por dois e arredondado para baixo (mínimo de uma cocção). As cocções devem ser de um nível que você é capaz de preparar e utilizar.
	\par Por exemplo, se você é um doutor da praga de 3º nível, você possui quatro cocções diárias de 1º nível, e duas de 2º nível. Com uma inteligência de 16, sua lista de cocções preparadas do dia pode incluir seis cocções de 1º ou 2º nível em qualquer combinação. Se você preparar a cocção \textit{bálsamo cicatrizante}, você pode utilizá-la gastando um dos seus usos diários de 1º nível ou um de 2º nível. Utilizar a cocção não a remove da sua lista de cocções preparadas.
	\par Você pode mudar a sua lista de cocções preparadas quando termina um descanso longo. Preparar uma nova lista de cocções requer que você invista tempo estudando seu livro de fórmulas e memorizando os ingredientes e passos do preparo, bem como os métodos e cuidados de uso da cocção: pelo menos 1 minuto por nível de cocção de cada cocção em sua lista.
	
	\subsubsection{Atributo de Cocção}
	Inteligência é seu atributo chave para preparo e uso das cocções, afinal elas são fruto de estudo e trabalho mental. Use seu modificador de inteligência quando você estiver determinando a CD de uma cocção que você utilizar e quando tiver de fazer uma rolagem de ataque quando estiver atacando com uma.
	
	{\footnotesize
	\textbf{CD do teste de resistência à cocção} = 8 + seu bônus de proficiência + seu modificador de Inteligência\\
	\textbf{Modificador de ataque com cocção} = seu bônus de proficiência + seu modificador de Inteligência
	}

	\subsubsection{Aprendendo Fórmulas de 1º Nível em Diante}
	Cada vez que você ganhar um nível de doutor da praga, você pode adicionar uma fórmula de cocção de sua escolha ao seu livro de fórmulas. Essa fórmula deve ser de um nível que você é capaz de preparar e utilizar. Nas suas aventuras, é possível que você encontre outras fórmulas que pode adicionar ao seu livro (ver o adendo "Seu Livro de Fórmulas").
	
	\subsubsection{Especialização Médica}
	Quando você alcance o 2º nível, você escolhe uma especialização médica, que potencializa e concentra suas habilidades de doutor da praga num campo específico. Sua escolha fará com que você receba habilidades no 2º nível, e novamente no 5º, no 7º, no 9º, no 11º, no 13º, no 15º, no 17º, e no 20º nível.
	
	\subsubsection{Melhoria de Atributo}
	Quando você alcanca o 4º nível, e novamente no 8º, no 12º, no 16º e no 19º nível, você pode aumentar um dos seus atributos em +2, ou dois de seus atributos em +1. Você não pode utilizar essa característica para aumentar um dos seus atributos acima de 20.
	
	\section{Especializações Médicas}
	Os doutores da praga costumam ter métodos e ideais semelhantes, porém diversos. Para realmente alcançar um nível elevado de maestria, o doutor da praga se vê obrigado a selecionar um campo de estudo mais estreito onde focará seus esforços de pesquisa. A especialização médica escolhida define que campo é esse.
	
	\subsection{Mutalista}
	\begin{quotebox}
		"O que não te mata, te deixa mais estranho", Jeremias Olho-Torto, Mutalista
	\end{quotebox}

	O Mutalista acredita que a maneira mais eficaz de combater a doença e atingir a apoteose não está meramente nas cocções, produtos externos, mas sim no potencial não explorado do próprio corpo. Na sua pesquisa, o Mutalista se utiliza de cobaias diversas, melhoradas através de cirurgias, ingestão de mutagênicos, e experimentos do gênero.
	
	\subsubsection{Mutante de Estimação}
	A partir de quando você adotar a especialização médica Mutalista, no 2º nível, você se torna capaz de criar e treinar uma criatura modificada para assisti-lo e protegê-lo. Escolha uma criatura de tamanho Médio ou menor, e com classe de desafio 1/4 ou inferior, para ser seu companheiro animal. Adicione seu bônus de proficiência na CA da criatura, rolagens de acerto e dano de ataques, bem como quaisquer testes de resistência ou perícias em que a criatura seja proficiente. Além disso, a criatura recebe uma das mutações da tabela de Mutações Primárias à sua escolha.
	\par A criatura te obedece tão bem quanto possível. Ela age no seu turno, apesar de não fazer nenhum ação, a menos que você a comande. No seu turno, você pode dar instruções para a criatura se mover (sem gastar nenhuma ação). Você pode usar sua ação para comandar a criatura a tomar a ação \textit{Atacar}, ou \textit{Correr}, ou \textit{Desengajar}, ou \textit{Esquivar} ou \textit{Ajudar}.
	\par Caso a criatura morra, você pode realizar uma cirurgia mutalística em um outro espécime para substituí-la. A cirurgia leva 8 horas.
	
	\header{Mutações Primárias}
	{\footnotesize
		\begin{dndtable}
			\label{tab:mutacoes_primarias}
			\textbf{Mutação} & \textbf{Descrição} \\
			Crescimento Desenfreado & A criatura aumenta em um sua classe de tamanho. Seus dados de vida mudam de acordo, e ela recebe +2 de Constituição.\\
			Reflexos Aprimorados & A criatura recebe proficiência em testes de resistência de Destreza +2 em CA \\
			Hiper Densidade Muscular & A criatura recebe proficiência em Atletismo e +2 de Força. \\
			Sentidos Aprimorados & A criatura recebe proficiência em Percepção e +2 de Sabedoria. \\
			Mimetismo & A criatura se torna proficiente em testes de Furtividade e tem vantagem em ataques contra criaturas que estejam em combate corpo-a-corpo com um de seus aliados. Seus ataques contra criaturas surpresas sempre contam como críticos. \\
			Hiper Metabolismo & A criatura recebe proficiência em testes de resistência de Constituição e +2 dados de vida. \\
			Mega Glândulas Adrenais & A criatura recebe +2 em rolagens de ataque e pode se mover uma distância igual a seu deslocamento quando realizar a ação \textit{Atacar}. O deslocamento pode ser dividido de qualquer forma antes ou depois do ataque, como numa ação de movimento normal.
		\end{dndtable}
	}
		
	\section{Pílulas}
	\label{sec:pilulas}
	Pílulas são feitas a partir de compostos mais simples, e podem ser produzidas em grandes quantidades. Um doutor da praga tem acesso a um número ilimitado de pílulas. Segue uma lista de diversas pílulas.
	
	\spellheader%
	{Pílula de Fumaça}
	{Pílula}
	{1 ação}
	{18m (12 hex)}
	{S}
	{Três rodadas}
	Você lança uma pílula de fumaça e ela preenche de fumaça o espaço num raio de 3m (2 hex em cada direção).
	
	\spellheader%
	{Pílula Corrosiva}
	{Pílula}
	{1 ação}
	{18m (12 hex)}
	{S}
	{Três rodadas}
	Você lança uma pílula num alvo, ou num par de alvos que estejam no máximo a 1.5m um do outro (1 hex), que explode espalhando ácido. Os alvos devem ser bem sucedidos num teste de resistência de Destreza, ou tomar 1d6 de dano. O dano dessa pílula aumenta quando você atinge o 5º nível (2d6), o 11º (3d6) e o 17º (4d6).
	
	\section{Cocções}
	Cocções são preparos especiais que um doutor da praga pode preparar. Devido a complexidade e a carga de trabalho mental necessários, o doutor da praga pode preparar apenas algumas dessas cocções por dia, conforme a \hyperref[tab:doutor_da_praga]{\textbf{tabela}} da classe.
	Segue uma lista de diversas cocções.
	
	\spellheader%
	{Bálsamo Cicatrizante}
	{Cocção de 1º nível}
	{1 ação}
	{Toque}
	{S}
	{Instantâneo}
	Você ou um aliado regenera pontos de vida iguais a 1d8 + seu modificador de inteligência.
	\subparagraph{Níveis superiores} Para cada nível de cocção acima do primeiro, essa cocção regenera +1d8 de vida.
	
\end{document}
