\documentclass[letterpaper,twocolumn,openany]{dndbook}

\usepackage[portuguese]{babel}

\usepackage[utf8]{inputenc}
\usepackage[singlelinecheck=false]{caption}
\usepackage{lipsum}
\usepackage{listings}
\usepackage{shortvrb}
\usepackage{stfloats}
\usepackage{multirow}
\usepackage[absolute,overlay]{textpos}

\newcolumntype{L}[1]{>{\hsize=#1\hsize\raggedright\arraybackslash}X}%
\newcolumntype{R}[1]{>{\hsize=#1\hsize\raggedleft\arraybackslash}X}%
\newcolumntype{C}[1]{>{\hsize=#1\hsize\centering\arraybackslash}X}%

\begin{document}
	\section{Doutor da Praga}
	
	Um chapéu escuro cobrindo a cabeça, uma máscara bicuda escondendo seu rosto. A capa, luvas e botas, completam o serviço de tornar a figura completamente enigmática e anônima.
	
	O Doutor da Praga é uma figura sinistra, e quase sempre sua presença é um mal sinal. Isso é claro, na mente dos ignorantes que nada sabem a respeito da Ciência e mais particularmente da Medicina.
	
	No meio de um mundo tomado pela magia, superstição e  ignorância, você é um dos poucos dotados de uma dose saudável de ceticismo, e também de perspicácia o bastante para explorar a miríade de segredos que a anatomia do corpo mantém trancados a sete chaves.
	
	Claro, devemos admitir que nem todos os doutores tem boas intenções. Pesquisas nem sempre tem um propósito nobre, e os métodos utilizados nem sempre podem ser considerados "éticos". Mas mesmo assim, não se pode culpar a Ciência pelo uso que se faz dela.
	
	Além disso, nem mesmo os bons doutores costumam fazer o tipo convencional. Coisas estranhas podem acontecer quando se passa tempo o bastante dissecando cadáveres e misturando reagentes num laboratório escuro.
	
	\subsection{Poder da Ciência}
	Os doutores da praga acreditam no potencial da inteligência humana acima de tudo. E para eles, a forma mais eficaz de desenvolver esse potencial é através da ciência.
	
	Suas habilidades podem parecer sobrenaturais ou mágicas para alguns, mas para eles, tudo pode ser explicado em termos científicos, mesmo que nem todos sejam capazes de compreender.
	
	Alguns doutores da praga compartilham uma curiosidade também pelos conhecimentos arcanos, enquanto outros acham que se trata de uma perda de tempo. Outros acreditam que a magia é para poucos, e que a ciência é muito mais apropriada para a maioria das mentes. Outros ainda, acreditam que magia e ciência são dois lados da mesma moeda, abordagens diferentes para compreender o mundo.
	
	De toda forma, os doutores da praga preferem colocar suas fichas nas explicações científicas do que na superstição transmitida pelos seus ancestrais ignorantes.

	\onecolumn
	\begin{textblock*}{5cm}(14cm, 1.4cm) % {block width} (coords)
		\footnotesize \sffamily \bfseries -Cocções Diárias por Nível-
	\end{textblock*}
	\header{Doutor da Praga}
	{\footnotesize
	\begin{dndtable}[R{0.25} C{1.5} L{3} C{1.5} R{0.25} R{0.25} R{0.25} R{0.25} R{0.25} R{0.25} R{0.25} R{0.25} R{0.25}]
		\textbf{Nível} & \textbf{Bônus de Proficiência} & \textbf{Características} & \textbf{Pílulas Conhecidas} & \textbf{1º} & \textbf{2º} & \textbf{3º} & \textbf{4º} & \textbf{5º} & \textbf{6º} & \textbf{7º} & \textbf{8º} & \textbf{9º} \\
		1 & +2 & Letra de Médico, Química  & 2 & 2 & - & - & - & - & - & - & - & - \\
		2 & +2 &  & 2 & 3 & - & - & - & - & - & - & - & - \\
		3 & +2 &  & 2 & 4 & 2 & - & - & - & - & - & - & - \\
		4 & +2 & Melhoria de Atributo & 3 & 4 & 3 & - & - & - & - & - & - & - \\
		5 & +3 &  & 3 & 4 & 3 & 2 & - & - & - & - & - & - \\
		6 & +3 &  & 3 & 4 & 3 & 3 & - & - & - & - & - & - \\
		7 & +3 &  & 3 & 4 & 3 & 3 & - & - & - & - & - & - \\
		8 & +3 & Melhoria de Atributo & 3 & 4 & 3 & 3 & - & - & - & - & - & - \\
		9 & +4 &  & 3 & 4 & 3 & 3 & - & - & - & - & - & - \\
		10 & +4 &  & 4 & 4 & 3 & 3 & - & - & - & - & - & - \\
		11 & +4 & & 4 & 4 & 3 & 3 & - & - & - & - & - & - \\
		12 & +4 & Melhoria de Atributo & 4 & 4 & 3 & 3 & - & - & - & - & - & - \\
		13 & +5 & & 4 & 4 & 3 & 3 & - & - & - & - & - & - \\
		14 & +5 & & 4 & 4 & 3 & 3 & - & - & - & - & - & - \\
		15 & +5 & & 4 & 4 & 3 & 3 & - & - & - & - & - & - \\
		16 & +5 & Melhoria de Atributo & 4 & 4 & 3 & 3 & - & - & - & - & - & - \\
		17 & +6 & & 4 & 4 & 3 & 3 & - & - & - & - & - & - \\
		18 & +6 & & 4 & 4 & 3 & 3 & - & - & - & - & - & - \\
		19 & +6 & Melhoria de Atributo & 4 & 4 & 3 & 3 & - & - & - & - & - & - \\
		20 & +6 & & 4 & 4 & 3 & 3 & - & - & - & - & - & - \\
	\end{dndtable}
	}
	\twocolumn
	
	\subsection{Apoteose Antrópica}
	
	Os doutores da praga acreditam na Ciência como um meio dos mortais deixarem para trás as mazelas que os afligem: a fome, a peste e a velhice. Para esse fim, basta que a Ciência avance o suficiente.
	
	É melhor do que fazer barganhas com seres sombrios ou desperdiçar os melhores dias da vida rogando súplicas a uma estátua glorificada qualquer.
\end{document}
